\documentclass[10pt,a4paper]{article}
\author{Sebastian Markgraf}
\title{Sicherheit}

\usepackage[utf8]{inputenc}
\usepackage[ngerman]{babel}
\usepackage{multicol}
\usepackage{amsmath}
\usepackage{amssymb}
\usepackage{mathtools}
\usepackage[activate={true,nocompatibility},final,tracking=true,kerning=true,spacing=true,factor=1100,stretch=10,shrink=10]{microtype}
\usepackage[a4paper, total={6in, 8in}]{geometry}
\usepackage{mathrsfs}

\def\realnumbers{{\rm I\!R}}
\def\polynomials{{\rm I\!P}}

\newcommand{\rom}[1]{\uppercase\expandafter{\romannumeral #1\relax}}
\newcommand{\norm}[1]{\lVert#1\rVert}
\renewcommand{\arraystretch}{1.5}

\begin{document}
	\pagenumbering{Roman}
	{\let\newpage\relax\maketitle}
	\tableofcontents
	\newpage
	\pagenumbering{arabic}
	\setcounter{page}{1}

        \section{Grundlagen}
        \subsection{Kerckhoffs' Prinzip}
        Die Sicherheit eines Verfahrens beruht auf der \textbf{Geheimhaltung des Schlüssels}
        anstatt auf der \textbf{Geheimhaltung des Verschlüsselsungsverfahrens}.
        
	\section{Symmetrische Verschlüsselung}
        \subsection{Betriebsmodi}
        \begin{itemize}
        \item ECB - Electronic Code Book Mode - Verschlüssele jeden Block einzeln
        \item CBC - Cipher Block Chaining Mode - Verknüpfe mit Chiffrat des vorherigen
        \item CTR - Counter Mode - Verschlüssele IV + i und addiere XOR zur Nachricht
        \end{itemize}

        \section{Asymmetrische Verschlüsselung}
        \subsection{RSA}
        \textbf{Schlüsselgenerierung}
        \begin{itemize}
          \item Wähle Primzahlen \(P \neq q\) zufällig
          \item berechne \(N = P \cdot Q, \varphi(N) = (P-1)(Q-1)\)
          \item wähle \(e\) zufällig aus \(\{3,...,\varphi(N)-1\}\), sodass \(\gcd(e, \varphi(N)) = 1\)
          \item berechne \(d \coloneqq e^{-1} \bmod \varphi(N)\)
          \item gib \(\mathit{pk} \coloneqq (N, e)\) und \(\mathit{sk} \coloneqq (N, d)\) aus
        \end{itemize}
        \textbf{Standardangreifer}\\
        \textit{Signatur (EUF-CMA)}:\\
        Wähle \(\sigma\) und setze \(m = \sigma^e\)
        \subsection{ElGamal}
        \textbf{Schlüsselerzeugung}
        \[\mathit{pk} = (\mathbb{G},g,h)\]
        \[\mathit{sk} = (\mathbb{G},g,x)\]
        \textbf{Ver- und Entschlüsselung}
        \[\textsc{Enc}(\mathit{pk}, M) = (g^y, h^yM)\]
        \[\textsc{Dec}(\mathit{sk}, g^y, C) = \frac{C}{(g^y)^x}\]
        \textbf{Informationsdichte}\\
        Bei optimaler Codierung hat ein ElGamal Chiffrat \(\log_2 | \mathbb{Z}_p^{\times}| = \log_2(p-1)\)
        Bit Informationen. Da Chiffrat aus 2 Gruppenelementen, ist das Verhältnis von Nutzinformationen zu Chiffratlänge max 1/2. \\
        \textbf{Signieren}\\
        Wähle eine zufällige, invertierbare Zahl \(e \in \{1,..., p-1\}\), wobei \(p - 1 = |\mathbb{G}|\)
        Berechne:
        \begin{align*}
          a &\coloneqq g^e \in \mathbb{G} \\
          b &\coloneqq (M - a \cdot x) \cdot e^{-1} \mod |\mathbb{G}| \Leftrightarrow m = ax + eb\\
          \textsc{Sig}(sk, M) &\coloneqq (a,b)
        \end{align*}
        Verifizieren der Signatur
        \begin{align*}
          v_1 &= (g^x)^a \cdot a^b\\
          v_2 &= g^M\\
          \textsc{Ver}(pk,\sigma, M) &= 1 \Leftrightarrow v_1 = v_2
        \end{align*}
        \textbf{Standardangreifer}\\
        \textit{Verschlüsselung}:\\
        Bekanntes \(C \coloneqq (9,1)\) gegeben.\\
        Berechne \(C* \neq C\) aber \(M* = M\).
        \begin{enumerate}
        \item Wähle zufälligegs \(z \in \mathbb{Z}_{11}\), z.B. \(y=1\)
        \item Berechne \(C'\) von 1: \(C' = (g^y, g^xy \cdot M) = (3^y, 9^y \cdot M)\)
        \item Multipliziere die Chiffrate: \(C* = C \cdot C' = (5,9)\) von \(1 \cdot M\)
        \end{enumerate}
        \textit{Signatur (EUF-CMA)}:
        \begin{enumerate}
        \item Nullsignatur: Wähle \((g^x, -g^x)\) als Signatur für \(m=0\)
          \item Unsinssnachricht: Berechne \(a \coloneqq g^zg^x\) dann \((a, -a)\) Signatur für \(m=-az\)
        \end{enumerate}

        \subsection{DES}
        \textbf{Meet-In-The-Middle}
        % TODO Meet-In-The-Middle allgemein aufschreiben. Wird in 2013(2) und 2014(1) abgefragt 
        \section{Hashverfahren}
        \subsection{HMAC}
        kurz für \textbf{Keyed-Hash Message Authentication Code} \\
        \[\mathit{HMAC}_{K}(M) = H((K \oplus \mathit{opad}) || H((K \oplus \mathit{ipad}) || M))\]

        \section{Zero-Knowledge}
        \subsection{Commitments}
        \textbf{Eigenschaften}
        \begin{itemize}
        \item \textit{hiding} - \(\textsc{Com}(M; R)\) verrät zunächst keinerlei Informationen über M
        \item \textit{binding} - \(\textsc{Com}(M; R)\) legt den Ersteller des Commitment auf \(M\) fest, d.h.
          der Ersteller der Nachricht kann später nicht glaubhaft behaupten, dass \(M' \neq M\) zur Erstellung
          Commitment
        \end{itemize}
        
        \section{Rechentricks}
        \subsection{Modulo}
        \textbf{Handhabbare Repäsentanten}
        \[51^2 \bmod 59 = (59-8)^2 \bmod 59 = (-8)^2 \bmod 59 = 64 \bmod 59 = 5 \bmod 59\]
        \textbf{Reduzierung des Exponenten modulo der Gruppenordnung}
        \[3^{60} \bmod 59 = 3^{60 \bmod 58} \bmod 59 = 3^2 \bmod 59\]
        Allgemein: Ordnung ist \(\varphi(N)\)
\end{document}
