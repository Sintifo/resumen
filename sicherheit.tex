\documentclass[10pt,a4paper]{article}
\author{Sebastian Markgraf}
\title{Sicherheit}

\usepackage[utf8]{inputenc}
\usepackage[ngerman]{babel}
\usepackage{multicol}
\usepackage{amsmath}
\usepackage[a4paper, total={6in, 8in}]{geometry}
\usepackage{mathrsfs}

\def\realnumbers{{\rm I\!R}}
\def\polynomials{{\rm I\!P}}

\newcommand{\rom}[1]{\uppercase\expandafter{\romannumeral #1\relax}}
\newcommand{\norm}[1]{\lVert#1\rVert}
\renewcommand{\arraystretch}{1.5}

\begin{document}
	\pagenumbering{Roman}
	{\let\newpage\relax\maketitle}
	\tableofcontents
	\newpage
	\pagenumbering{arabic}
	\setcounter{page}{1}

	\section{Symmetrische Verschlüsselung}
        \subsection{Betriebsmodi}
        \begin{itemize}
        \item ECB - Electronic Code Book Mode - Verschlüssele jeden Block einzeln
        \item CBC - Cipher Block Chaining Mode - Verknüpfe mit Chiffrat des vorherigen
        \item CTR - Counter Mode - Verschlüssele IV + i und addiere XOR zur Nachricht
        \end{itemize}
\end{document}
