\documentclass[10pt,a4paper]{article}
\author{Sebastian Markgraf}
\title{Sicherheit}

\usepackage[utf8]{inputenc}
\usepackage[ngerman]{babel}
\usepackage{multicol}
\usepackage{amsmath}
\usepackage[a4paper, total={6in, 8in}]{geometry}
\usepackage{mathrsfs}

\def\realnumbers{{\rm I\!R}}
\def\polynomials{{\rm I\!P}}

\newcommand{\rom}[1]{\uppercase\expandafter{\romannumeral #1\relax}}
\newcommand{\norm}[1]{\lVert#1\rVert}
\renewcommand{\arraystretch}{1.5}

\begin{document}
	\pagenumbering{Roman}
	{\let\newpage\relax\maketitle}
	\tableofcontents
	\newpage
	\pagenumbering{arabic}
	\setcounter{page}{1}

        \section{Grundlagen}
        \subsection{Kerckhoffs' Prinzip}
        Die Sicherheit eines Verfahrens beruht auf der \textbf{Geheimhaltung des Schlüssels}
        anstatt auf der \textbf{Geheimhaltung des Verschlüsselsungsverfahrens}.
        
	\section{Symmetrische Verschlüsselung}
        \subsection{Betriebsmodi}
        \begin{itemize}
        \item ECB - Electronic Code Book Mode - Verschlüssele jeden Block einzeln
        \item CBC - Cipher Block Chaining Mode - Verknüpfe mit Chiffrat des vorherigen
        \item CTR - Counter Mode - Verschlüssele IV + i und addiere XOR zur Nachricht
        \end{itemize}

        \section{Asymmetrische Verschlüsselung}
        \subsection{RSA}
        \subsection{ElGamal}

        \section{Hashverfahren}
        \subsection{HMAC}
        kurz für \textbf{Keyed-Hash Message Authentication Code} \\
        \[\mathit{HMAC}_{K}(M) = H((K \oplus \mathit{opad}) || H((K \oplus \mathit{ipad}) || M))\]

        \section{Rechentricks}
        \subsection{Modulo}
        \textbf{Handhabbare Repäsentanten}
        \[51^2 \bmod 59 = (59-8)^2 \bmod 59 = (-8)^2 \bmod 59 = 64 \bmod 59 = 5 \bmod 59\]
        \textbf{Reduzierung des Exponenten modulo der Gruppenordnung}
        \[3^{60} \bmod 59 = 3^{60 \bmod 58} \bmod 59 = 3^2 \bmod 59\]
        Allgemein: Ordnung ist \(\varphi(N)\)
\end{document}
